\begin{abstract}
Our approach is one of integrating the HSR into existing robotics research of several collaborating laboratories here at the University of Texas at Austin. These laboratories focus on artificial intelligence, machine learning, natural language, control, and human-robot interaction. Experiments are performed in simulated home environments where robots provide services to and interact with human users and in a live deployment known as the Building-Wide Intelligence (BWI) Project in the Gates-Dell Computer Science Complex. Incorporating our RoboCup effort with our existing research, we are able to deploy relevant software components for use in the competition. We also are reusing software developed for the competition as it is integrated our research projects. Collaboration on RoboCup also encourages our critical mass of expert researchers to further collaborate in our research endeavors. Significant portions of the code developed by our group for competition will be released as components of the BWI infrastructure code as open source software through ROS.org. Relevant current work by our groups involves research in natural language processing, multimodal language learning, learning from demonstration, human activity recognition, planning, control, and probabilistic and symbolic reasoning.%In the past year, significant work has been performed on domestic tasks such as storing groceries, and language tasks such as natural language processing and the performance of natural language tasks from speech.

%In your abstract, please state your main research line and your achievements of this year (on which problem or set of problems are you focusing all the team efforts). Tell why this research is important, how are you approaching to the problem solution and which results do you expect to obtain.

\end{abstract}

